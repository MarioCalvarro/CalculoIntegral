\section{Medida}
El concepto fundamental de esta sección es comprender qué entendemos por medida y que propiedades tiene, así como demostrar que la medida de Lebesgue es un buen instrumento de medida para trabajar en $\mathbb{R}^n$.

\begin{defi}[Medida]
Sea una $\sigma$-álgebra $\left( X, M \right)$, una \textbf{medida positiva} será $\mu: M \rightarrow \overline{\mathbb{R}^+} \cup \{0\}$ tal que:
\begin{itemize}
    \item $\mu\left( \emptyset \right) = 0$
    \item $\mu\left( \bigcup_{j = 1}^{\infty} A_j \right) = \sum_{i=1}^{\infty} \mu \left( A_j \right)\quad A_j \cap A_i = \emptyset,\ \forall i \neq j$. (\textbf{$\sigma$ aditividad})
\end{itemize}
\end{defi}

\begin{defi}[Rectángulo]
Definimos un rectángulo en $\mathbb{R}^n$ como el conjunto;
$$Q = \left[ a_1, b_1 \right] \times \ldots \times \left[ a_n, b_n \right] \subset \mathbb{R}^{n}$$
Además, definimos como \textbf{volumen} de $Q$ a 
$$v\left( Q \right) = \left( b_n - a_n \right) \cdots \left( b_1 - a_1 \right)$$
\end{defi}

\begin{defi}[Medida exterior de Lebesgue]
Sea $A \subset \mathbb{R}^{n}$, definimos como \textbf{medida exterior} de $A$ a:
$$
\mu^{*}(A) = \inf \sum_{k=1}^{\infty} v\left( Q_k \right); \text{ donde } \{Q_k\}_{k=1}^{\infty} \text{ y } A \subset \bigcup_{k = 1}^{\infty} Q_k
$$ 
\end{defi}

\begin{obs}
Podemos restringir las familias $\{Q_k\}$ a aquellas que tengan $\mathrm{diam}\ Q_k < \delta$ para un cierto delta y la definición no cambia. Esto es así porque, en el fondo, el conjunto de sumatorios de volúmenes que estamos escogiendo es el mismo, ya que para familias con algún $Q_k$ de diámetro mayor, éste se puede dividir en varios rectángulos más pequeños cuyo volumen suman el de $Q_k$ pero cuyo diámetro es menor que dicho delta.
\end{obs}

\begin{defi}[Medida nula]
Decimos que $A$ tiene medida nula si podemos encontrar recubrimientos de $A$ con sumatorio de volúmenes tan pequeños como queramos, es decir:
$$\mu^{*}\left( A \right) = 0 \Leftrightarrow \forall \varepsilon > 0 \ \exists \{Q_k\} \text{ rec. de } A: \sum_{k=1}^{\infty} v\left( Q_k \right) < \varepsilon $$  
\end{defi}

\begin{ej}
	\begin{enumerate}
       \item Si $x_0 \in \mathbb{R}^{n}$, entonces $\forall \varepsilon > 0, \exists k \in \mathbb N: \ v\left( Q\left( x_0 \right)  \right) < \frac{\varepsilon}{2^{k}}$, luego es de medida nula.
       \item Si $N$ numerable, entonces podemos escribir $N = \{x_k\}$, para cada $x_k$ podemos encontrar $Q_k\left( x_k \right): v\left( Q_k\left( x_k \right)  \right) < \frac{\varepsilon}{2^{k}}$ que lo contenga, luego se tiene que es de medida nula.
    \end{enumerate}
\end{ej}
    
\begin{prop}
La medida exterior de Lebesgue cumple las siguientes propiedades:
\begin{itemize}
\item \textbf{Invariante por traslaciones}:
$$A \subset \mathbb{R}^{n}, \ c \in \mathbb{R}^{n} \Rightarrow \mu^{*}\left( c + A \right) = \mu^{*}\left( A \right) $$ 
\item \textbf{Covariante con respecto a la inclusión}:
$$A \subset B \subset \mathbb{R}^{n} \Rightarrow \mu^{*}\left( A \right) \le \mu^{*}\left( B \right) $$
\item \textbf{Semiaditiva}:
$$A,\ B \in \mathbb{R}^{n} \Rightarrow \mu^{*}\left( A \cup B \right) \le \mu^{*}\left( A \right) + \mu^{*}\left( B \right)$$
\end{itemize}
\end{prop}
\begin{demo}
\begin{itemize}
\item Trivial

\item Si $\mu^{*}\left( B \right) = +\infty$ es trivial, luego suponemos que no.

Sea $\{Q_k\}$ un recubrimiento de $B$, como $A \subset B \Rightarrow \{Q_k\}$ es un recubrimiento de A, luego\footnote{Puesto que si eres menor que todos los elementos de un conjunto eres menor o igual que su ínfimo.}
$$\forall \{Q_k\}: \mu^{*}\left( A \right) < \sum_{k=1}^{\infty} v\left( Q_k \right) \Rightarrow \mu^{*}\left( A \right) \le \mu^{*}\left( B \right) $$

\item Si $\mu^{*}\left( A \right) = +\infty$ o $\mu^{*}\left( B \right) = +\infty$ es trivial. 
 
Suponiendo ambos finitos, entonces para $\varepsilon > 0$ se tiene que:
$$\begin{cases}
\exists \{Q_k\} \text{ rec. de A: } \sum_{k=1}^{\infty} v\left( Q_k \right) < \mu^{*}\left( A \right) + \frac{\varepsilon}{2}  \\
\exists \{R_k\} \text{ rec. de B: } \sum_{k=1}^{\infty} v\left( R_k \right) < \mu^{*}\left( A \right) + \frac{\varepsilon}{2} 
\end{cases}
$$
Consideremos $ \{Q_k, R_k\} = \{S_j\}$ donde $S_j = \begin{cases}
Q_{\frac{j}{2}},\ j \text{ par} \\
R_{\frac{j+1}{2}},\ j \text{ impar} 
\end{cases}$. Tenemos trivialmente que $\{S_j\}$ es rec. de $A\cup B$, luego:
$$\mu^{*}\left( A\cup B \right) \le \sum_{j=1}^{\infty} v\left( S_j \right) = \sum_{k=1}^{\infty} v\left( Q_k \right) + \sum_{k=1}^{\infty} v\left( R_k \right) < \mu^{*}\left( A \right) + \mu^{*}\left( B \right) + \varepsilon$$
Y como dicha desigualdad es para cualquier $\varepsilon > 0$ se tiene la desigualdad del enunciado.
\end{itemize}
\end{demo}

\begin{prop}
En cualquier rectángulo, la medida exterior y su volumen coinciden, esto es:
$$\forall Q \subset \mathbb{R}^n : v\left( Q \right) = \mu^{*}\left( Q \right)$$
\end{prop}
\begin{demo}
    \begin{itemize}
        \item $ \mu^*\left( Q \right) \le v\left( Q \right) $:
        
        Tomamos $\varepsilon > 0$ y consideramos la familia $ \{Q_k\}$ recubrimiento de $Q$ donde $Q_1 = Q$ y $\forall k \geq 2 : v(Q_k) < \varepsilon/2^{k}$. De este modo, $\{Q_k\} $ es rec. de $Q$ y además
$$\sum_{k=1}^{\infty} v\left( Q_k \right) = v\left( Q \right) + \sum_{k=2}^{\infty} v\left( Q_k \right) < v\left( Q \right) + \sum_{k=2}^{\infty} \frac{\varepsilon}{2^{k}} < v\left( Q \right) + \varepsilon$$
		Por tanto, por ser ínfimo y ser para todo epsilon, se tiene:
		$$\mu^*\left( Q \right) \le v\left( Q \right) + \varepsilon \Rightarrow \mu^*\left( Q \right) \le v\left( Q \right)$$
        \item $v\left( Q \right) \le \mu^*\left( Q \right)$:
        
        Observamos que $ \overline{Q}$ es la unión de las caras de $Q$ que denotaremos por $C_i$, por tanto:
        \begin{align*}
        v\left( Q \right) = v\left( \overline{Q} \right)  && \mu^{*}\left( Q \right) \le \mu^{*}\left( \overline{Q} \right)
        \end{align*}
        $$\mu^{*}\left( \overline{Q} \right) = \mu^{*}\left( Q \cup \left( C_1, \ldots, C_m \right)  \right) \le \mu^{*}\left( Q \right) + \mu^{*}\left( C_1 \right) + \ldots + \mu^{*}\left( C_m \right) = \mu^{*}\left( Q \right)$$
        Luego, concluimos que un cubo cualquiera tiene la misma medida que el mismo cubo, pero cerrado. Por tanto, podemos suponer que $Q$ es cerrado y por ser acotado en $\mathbb{R}^n$ es compacto. 
        
Si el volumen es menor que todos los posibles sumatorios, entonces tendrá que ser menor que el ínfimo, luego basta probar que $v\left( Q \right) \le \sum_{k=1}^{\infty} v\left( Q_k \right), \ \forall \{Q_k\} \ Q \subset \bigcup_{k \in \mathbb{N}} Q_k $.

Como $Q$ lo podíamos considerar compacto, se tiene que dado un recubrimiento $\{Q_k\}_{k=1}^{\infty}$:
$$Q \subset Q_1 \cup \ldots \cup Q_N \Rightarrow v\left( Q \right) \le v\left( Q_1 \right) +\ldots+v\left( Q_N \right) \le \sum_{k=1}^{\infty} v\left( Q_k \right)$$
\end{itemize}
\end{demo}

\begin{defi}
Definimos el \textbf{diámetro de un conjunto} como:
$$\mathrm{diam}\ A = \sup \{\vert \vert x - y \vert  \vert: \ x,y \in A\}$$
Y definimos la \textbf{distancia entre dos conjuntos} como:
$$d(A,B) = \inf \{||x-y||: x\in A, \ y \in B\}$$
\end{defi}

\begin{prop}
Para conjuntos a distancia positiva, la medida exterior tiene la propiedad aditiva:
$$\mathrm{d}\left( A, B \right) > 0 \Rightarrow \mu^*\left( A \cup B \right) = \mu^*\left( A \right) + \mu^*\left( B \right)$$
\end{prop}
\begin{demo}
Si $ \mu^*\left( A\cup B \right) = +\infty$ entonces es trivial, por tanto, podemos suponer que es finito.

Como conocemos una desigualdad, basta solo probar la otra. Tomamos $\delta > 0: \ \delta < \frac{\mathrm{d}\left( A, B \right)}{2}$ y $\forall \varepsilon > 0: \ \exists \{Q_k\} \text{ rectángulos }: \ \mathrm{diam}\ Q_k < \delta$ de forma que $A\cup B \subset \bigcup_{k \in \mathbb{N}} Q_k$, es decir, recubren y además $\sum_{k=1}^{\infty} v\left( Q_k \right) < \mu^*\left( A\cup B \right) + \varepsilon $.

Estos $Q_k$ tienen la propiedad de que, o bien $Q_k \cap A = \emptyset$, o bien $Q_k \cap B = \emptyset$, ya que si no fuese así existirían $a,b\in \mathbb{R}^n: d(a,b) < \mathrm{diam} \ Q_k < \delta < d(A,B)$ lo cual es absurdo. 

Podemos dividir el recubrimiento entonces en dos conjuntos:
$$ \begin{cases}
C:= \{Q_k : Q_k \cap A \neq \emptyset\} \text{ recubrimiento de } A \\
D:= \{Q_k : Q_k \cap B \neq \emptyset\} \text{ recubrimiento de } B
\end{cases}$$
Y no incluimos los que no cortan a ninguno puesto que esos sobran, en consecuencia:
$$\mu^*(A\cup B) +\varepsilon \geq \sum_{k=1}^{\infty} v\left( Q_k \right) \geq \sum_{C}^{\infty} v\left( Q_k \right) + \sum_{D}^{\infty} v\left( Q_k \right) \geq \mu^*\left( A \right) + \mu^*\left( B \right)$$
Y como es para todo epsilon, se tiene la desigualdad que nos faltaba.
\end{demo}

\begin{theo}[Medida exterior]
    La medida exterior de Lebesgue cumple las siguientes propiedades: 
    \begin{enumerate}
        \item $ \mu^*\left( \emptyset \right) = 0 $ 
        \item $A \subset B \Rightarrow \mu^*\left( A \right) \le \mu^*\left( B \right)$
        \item $ \mu^*\left( \bigcup_{k\in \mathbb{N}} A_k \right) \le \sum_{k=1}^{\infty} \mu^*\left( A_k \right)$
    \end{enumerate}
\end{theo}
 \begin{demo}
La propiedad 2 ya la tenemos demostrada, y la propiedad uno viene de aplicar la 2 a un conjunto numerable, puesto que el vacío pertenece a él y este es de medida nula. 

Para demostrar la 3, si $\exists k: \mu^*\left( A_k \right) = +\infty$ es trivial, luego suponemos que $ \mu^*\left( A_k \right) < +\infty$ y tomamos $ \varepsilon>0$, entonces para cada $A_k$ hay un recubrimiento que verifica:
$$ \forall k \in \mathbb N, \ \exists \{Q_j^k\}_{j=1}^\infty : A_k \subset \bigcup_{j \in \mathbb{N}} Q_j^{k} : \sum_{j=1}^{\infty} v\left( Q_j^k \right) < \mu^*\left( A_k \right) + \frac{\varepsilon}{2^k}
$$
De este modo, como cada $A_k$ está contenido en la unión sobre $j$ de un conjunto $\{Q_j^k\}$ para $k$ concreto, la unión de todos los $A_k$ está contenida en la unión sobre $k$ de dichos recubrimientos, es decir: 
$$\bigcup_{k \in \mathbb{N}} A_k \subset \bigcup_{k \in \mathbb{N}} \left(\bigcup_{j \in \mathbb{N}} Q_j^k  \right) = \bigcup_{j, k = 1} ^{\infty} Q_j^k$$
Como $\{Q_j^k\}$ es numerable, lo anterior es una unión numerable de conjuntos numerables,  entonces:
$$\mu^*\left( \bigcup_{k\in \mathbb{N}} A_k \right) \le \sum_{j, k} v\left( Q_j^k \right) = \sum_{k=1}^{\infty} \sum_{j=1}^{\infty} v\left( Q_j^k \right) < \sum_{k=1}^{\infty} \left( \mu^*\left( A_k \right) + \frac{\varepsilon}{2^k} \right) = \left( \sum_{k=1}^{\infty} \mu^*\left( A_k \right) \right) + \varepsilon $$
Y como es para todo $\varepsilon$ se tiene la desigualdad.
\end{demo}

\begin{obs}
La condición de $d(A,B) > 0$ para que la medida exterior de la unión sea la suma de las medidas exteriores no se puede relajar, es decir, existen conjuntos disjuntos en $\mathbb{R}^n $ con $d(A,B)>0$ para los cuales la medida de la suma no es la suma de las medidas.
$$\exists A, B: A\cap B = \emptyset\; \land \;\mu^*\left( A\cup B \right) < \mu^*\left( A \right) + \mu^*\left( B \right)$$
\end{obs}

\begin{defi}[Conjunto Medible]
Sea $A \subset \mathbb{R}^n$, se dice que es \textbf{medible} si y sólo si
$$\forall S \subset \mathbb{R}^n : \mu^*\left( S \right) = \mu^*\left( S\cap A \right) + \mu^*\left( S \cap A^c \right)$$
\end{defi}


\begin{theo}[De Caratheodory]
Los conjuntos medibles forman una $\sigma$-álgebra y la medida exterior de Lebesgue, es decir:
\begin{itemize}
\item $\emptyset \in F_\sigma$
\item $A\in F_ \sigma \Rightarrow A^c \in F_\sigma$
\item $\forall k \in \mathbb{N}: A_k \in F_\sigma \Rightarrow \bigcup_{k=1}^\infty A_k \in F_\sigma$
\end{itemize}
Además, la medida exterior de Lebesgue es $ \sigma$-aditiva cuando la restringimos a los conjuntos medibles, es decir:
$$\mu^*\left(\bigsqcup_{n\in \mathbb{N}}A_n\right) = \sum_{n=1}^\infty \mu^*(A_n)$$
Una vez restringida a la $\sigma-$álgebra, la medida exterior de Lebesgue cumple la definición de medida y se denota simplemente por $\mu$.
\end{theo}

\begin{prop}
Todo conjunto $A$ de medida nula es medible.
\end{prop}
\begin{demo}
La desigualdad $\mu^*\left( S \right) \le \mu^*\left( S\cap A \right) + \mu^*\left( S\cap A^c \right)$ siempre es cierta, luego hay que demostrar la otra, es decir:
$$ \mu^*\left( S\cap A \right) + \mu^*\left( S\cap A^c \right) \le \mu^*\left( S \right) $$
Pero esta es trivial, puesto que el primer sumando vale cero ya que es subconjuntos de $A$ y el segundo es subconjunto de $S$.
\end{demo}

\begin{prop}
Los rectángulos $Q$ son conjuntos medibles. 
\end{prop}
\begin{demo}
Siempre se cumple que $ \mu^*\left( S \right) \le \mu^*\left( S\cap Q \right) + \mu^*\left( S\cap Q^c \right)$, luego solo es necesario ver la otra ¿$ \mu^*\left( S\cap Q \right) + \mu^*\left( S\cap Q^c \right) \le \mu^*\left( S \right)$?

Tomemos un recubrimiento cualquiera de $S$, es decir:
$$\{Q_j\}: S \subset \bigcup_{j \in \mathbb{N}} Q_j$$
Observamos que $ \{Q_j \cap Q\} $ siempre son rectángulos que recubren a $S\cap Q$, luego:
$$\mu^*\left( S\cap Q \right) \le \sum_{j=1}^{\infty} v\left( Q_j \cap Q \right)$$
A su vez, $ \{Q_j \cap Q^c\}$ recubren a $S\cap Q^c$, pero no tienen porqué ser rectángulos. Sin embargo, sí son una unión finita de estos, es decir, para cada $Q_j \cap Q^c$ podemos escribir esta como $Q_j \cap Q^c = R_1 \cup \cdots \cup R_m$, por tanto, podemos recubrir mediante rectángulos como:
$$S \cap Q^c \subset \bigcup_{j=1}^\infty Q_j\cap Q^c = \bigcup_{j=1}^\infty \left(\bigcup_{i=1}^\infty  R_i^j \right)\Rightarrow \mu^*\left( S\cap Q^c \right) \le \sum_{i=1}^{\infty} v\left( R^j_i \right) $$

$$ R_1 \cup \cdots \cup R_m \cup \left( Q_j \cap Q \right) = Q_j \Rightarrow v\left( Q_j \right) = v\left( R_1 \right) +\ldots + v\left( R_m \right) + v\left( Q_j\cap Q \right) $$
En consecuencia, tenemos que:
$$\mu^*\left( S\cap Q \right) + \mu^*\left( S \cap Q^c \right) \le \sum_{j=1}^{\infty} v\left( Q_j \cap Q \right) + $$
$$+ \sum_{j=1}^{\infty} \left( v\left( Q_j\cap Q \right) + v\left( R_1^j \right) +\ldots + v\left( R_m^j \right) \right) = \sum_{j=1}^{\infty} v\left( Q_j \right)$$
\end{demo}

\begin{prop}
\begin{itemize}
\item Todo conjunto abierto es medible
\item Todo cerrado es medible
\item Las uniones numerables de cerrados (conjuntos $F_\sigma$) son medibles
\item Las intersecciones numerables de abiertos  (conjuntos $G_\delta$) son medibles
\end{itemize}
\end{prop}

\begin{prop}
Si $A\subset B$, ambos son conjuntos medibles y $\mu(A) < \infty$, entonces:
$$\mu(B\setminus A) = \mu(B) - \mu(A)$$
\end{prop}
\begin{demo}
Escribimos $B$ como unión disjunta de conjuntos de la forma $B=A\cup (B\setminus A)$, esto implica que $\mu(B) = \mu(A) + \mu(B\setminus A) \Rightarrow \mu(B\setminus A) =  \mu(B) - \mu(A)$.
\end{demo}

\begin{defi}[Sucesión creciente de conjuntos]
Decimos que una sucesión de conjuntos $\{A_k\}$ es creciente y lo denotamos por $\{A_k\}\uparrow$ si y sólo si $A_k \subset A_{k+1} : \forall k  \in \mathbb{N}$.
\end{defi}

\begin{prop}
Si tenemos una familia de conjuntos crecientes $\{A_k\}\uparrow$ medibles, entonces se tiene que:
$$\mu\left(\bigcup_{k=1}^\infty A_k\right) = \lim_{k\rightarrow \infty} \mu(A_k)$$
\end{prop}
%TODO: Corregir B_k
\begin{demo}
Si alguno tiene medida infinita se tiene trivialmente, luego podemos suponer que $\forall k .\in \mathbb{N}: \mu(A_k) < \infty$.

En primer lugar, vamos a construir la siguiente sucesión de conjuntos:
$$\begin{cases}
B_1 = A_1 \\
B_2 = A_2\setminus A_1 \\
\vdots \\ 
B_k = A_k \setminus A_{k-1}
\end{cases} \Rightarrow \bigcup_{k=1}^\infty B_k = \bigcup_{k=1}^\infty A_k \Rightarrow \mu\left(\bigcup_{k=1}^\infty B_k\right) = \mu\left(\bigcup_{k=1}^\infty A_k\right)$$
Sin embargo, como $\{B_k\}_{k=1}^\infty$ son disjuntos dos a dos, tenemos que:
$$\mu\left(\bigcup_{k=1}^\infty B_k\right) = \sum_{k=1}^\infty \mu(B_k) =\sum_{k=1}^\infty \mu(A_k\setminus A_{k-1}) =$$
$$\sum_{k=1}^\infty \mu(A_k)- \mu(A_{k-1}) = \lim_{k\rightarrow \infty} \mu(A_k) - \mu(\emptyset) = \lim_{k\rightarrow \infty}\mu(A_k)$$
\end{demo}

\begin{defi}[Sucesión decreciente de conjuntos]
Decimos que una sucesión de conjuntos $\{A_k\}$ es decreciente y lo denotamos por $\{A_k\}\downarrow$ si y sólo si $A_{k+1} \subset A_{k} : \forall k  \in \mathbb{N}$.
\end{defi}

\begin{prop}
Si tenemos una familia de conjuntos decrecientes $\{A_k\}\downarrow$ medibles y $\exists k \in \mathbb N : \mu(A_k) < \infty$, entonces se tiene que:
$$\mu\left(\bigcap_{k=1}^\infty A_k\right) = \lim_{k\rightarrow \infty} \mu(A_k)$$
\end{prop}
\begin{demo}
Completamente análoga a su homóloga anterior.
\end{demo}

\begin{theo}
Sea $A \subset \mathbb{R}^n$. Son equivalentes:  
\begin{enumerate}
\item $A$ es medible.
\item $\forall \varepsilon > 0,\ \exists G \supset A$ abierto $: \mu\left( G\setminus A  \right) < \varepsilon$  
\item $A = D \setminus N: D$ es $G_{\delta},\ \mu\left( N \right) = 0$  
\item $A = C \cup N: C$ es $F_{\sigma},\ \mu\left( N \right) = 0$  
\item $\forall \varepsilon > 0,\ \exists F \subset A$ cerrado $: \mu\left( A \setminus F \right) < \varepsilon$  
\end{enumerate}
\end{theo}
\begin{demo}
\begin{itemize}
\item $1 \Rightarrow 2$:

Supongamos primero que $A$ es acotado. Dado $\varepsilon > 0$, como $\mu\left( A \right) < +\infty$ por ser acotado, entonces $\exists \{Q_k\}_{k=1}^{\infty} : A \subset \displaystyle \bigcup_{k=1}^\infty Q_k$ de forma que $\sum_{k=1}^{\infty} v\left( Q_k \right) < \mu\left( A \right) + \varepsilon$. 

Como el volumen no depende de que los rectángulos sean abiertos o cerrados, consideramos los $Q_k$ rectángulos abiertos de modo que:
$$G = \bigcup_{k=1}^{\infty} Q_k \mbox{ abierto} : A\subset G$$
Como $\mu (A) < + \infty$, entonces podemos ver que:
$$\mu\left( G \setminus A \right) = \mu\left( G \right) - \mu\left( A \right) \le \sum_{k=1}^{\infty} \mu\left( Q_k \right) - \mu\left( A \right) < \varepsilon $$ 
En general, si $A$ no es acotado, definimos: 
$$A_k := \{x \in A: k - 1 \le \vert \vert x \vert  \vert < k \}$$

$$\begin{tikzpicture}
\begin{axis}[
axis line style={draw=none},
tick style={draw=none},
axis y line=middle,
axis x line=middle,
%xlabel=$x$,
%grid = both, %major/minor
xticklabels={},yticklabels={}
];

\addplot [domain=-1:1, samples=100, name path=f, thick, color=red!50]
        {sqrt(1 - x^2)};
        \addplot [domain=-1:1, samples=100, name path=f1, thick, color=red!50]
        {-sqrt(1 - x^2)};

\addplot [domain=-2:2, samples=100, name path=f, thick, color=red!50]
        {sqrt(4 - x^2)};
        \addplot [domain=-2:2, samples=100, name path=f1, thick, color=red!50]
        {-sqrt(4 - x^2)};
        
\addplot [domain=-3:3, samples=100, name path=f, thick, color=red!50]
        {sqrt(9 - x^2)};
        \addplot [domain=-3:3, samples=100, name path=f1, thick, color=red!50]
        {-sqrt(9 - x^2)};

\addplot [domain=-4:1.5, samples=100, name path=g, thick, color=black, very thick]
        {e^x};

\addplot[red!10, opacity=0.4] fill between [of=f and g, soft clip={domain=-4:1.03}];

\node[color=red,right] at (axis cs: -0.5,0.5) {$A_1$};
\node[color=red,right] at (axis cs: -0.5,1.5) {$A_2$};
\node[color=red,right] at (axis cs: -0.5,2.5) {$A_3$};

\node[color=black,right] at (axis cs: 1.03,3) {$A$};\\


\end{axis}

\end{tikzpicture}
$$
Los conjuntos $A_k$ son la intersección de $A$ con coronas disjuntas 2 a 2 concéntricas que descomponen $A$ como:
$$ A = \bigsqcup_{k=1}^{\infty}A_k$$ 
Si fijamos $\varepsilon > 0$ y como $\mu\left( A \right) < +\infty$, podemos aplicar el razonamiento anterior para acotados en cada $A_k$, luego:
$$\exists G_k \supset A_k \text{ abierto} : \mu\left( G_k \setminus A_k \right) < \frac{\varepsilon}{2^k}$$
Por tanto, si consideramos $G$ como la unión de los $G_k$, tenemos:
$$G = \bigcup_{k=1}^{\infty}G_k \text{ es un abierto : } G \supset A$$ 
Y se cumple que:
$$\mu\left( G \setminus A \right) = \mu\left( \left( \bigcup_{k \in \mathbb{N}}G_k  \right) \setminus A   \right) \le \mu\left(  \bigcup_{k \in \mathbb{N}} \left(G_k \setminus A_k  \right)  \right) \le \sum_{k=1}^{\infty} \mu\left( G_k - A_k \right) < \sum_{k=1}^{\infty} \frac{\varepsilon}{2^k} = \varepsilon $$ 

\item $2 \Rightarrow 3$:

$\forall k \in \mathbb{N}$, consideramos $G_k \subset A$ abierto tal que $\mu\left( G_k \setminus A \right) < \frac{1}{k}$, después definimos el conjunto:
$$D := \bigcap_{k=1}^{\infty}G_k \text{ que es } G_{\delta}:D \supset A$$ 
Como $A = D \setminus (D \setminus A)$, definimos entonces $N = D \setminus A$ y ocurre que:
$$\forall k \in \mathbb{N}: \mu\left( D \setminus A \right) \le \mu\left( G_k \setminus A \right) < \frac{1}{k} \Rightarrow \mu\left( N \right) = 0$$

\item $3 \Rightarrow 1$:

Tenemos $A = D \setminus N$, o lo que es lo mismo, que $A = D \cap N^c$. $D$ es medible por hipótesis, y $N^c$ es el complementario de un medible, que es medible\footnote{Por el teorema de Caratheodory}. Como las intersecciones de medibles son medibles, $A$ es medible.

\item $1 \Rightarrow 5$:

Sea $\varepsilon > 0$, como $A$ es medible, entonces $A^c$ también lo es y, por el apartado 2, tenemos que:
$$\exists G \supset A^c \text{ abierto : } \mu\left( G \setminus A^c \right) < \varepsilon$$ 
Sea $F := G^c \text{ cerrado}$, como  $A^c \subset G \Rightarrow F = G^c \subset \left( A^c \right)^c = A$, por tanto:
$$A \setminus F = \left( A^c \right)^c \setminus F  =\left( A^c \right)^c \cap F^c = G \cap \left( A^c \right)^c = G \setminus A^c$$
Luego, se tiene que:
$$\mu\left( A \setminus F \right) = \mu\left( G \setminus A^c \right) < \varepsilon$$

\item $5 \Rightarrow 4$: (Similar a $2 \Rightarrow 3$)\\
$\forall k \in \mathbb{N}$, consideramos $F_k \subset A : \mu\left( A \setminus F_k \right) < \frac{1}{k}$, después definimos el conjunto:
$$C := \bigcup_{k = 1}^{\infty}F_k  \text{ es } F_{\sigma} : \ C \subset A$$ 
Y la demostración es análoga a la implicación $2 \Rightarrow 3$.

\item $4 \Rightarrow 1$

De nuevo y por similitud con $3\Rightarrow 1$, $A = C \cup N$ es medible por ser unión de medibles.
\end{itemize}
\end{demo}
